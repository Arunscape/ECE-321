\documentclass[letterpaper,12pt]{article}
\synctex=1

% \usepackage[margin=1in]{geometry}
\usepackage{geometry}
% \usepackage{lipsum}
%
% \usepackage{hyperref}
%
% \usepackage{graphicx}


% \bibliographystyle{IEEEtran}

\title{ECE 321: Software Requirements Engineering \\ Assignment 1}
\author{Arun Woosaree \\ \\ XXXXXXX}


\begin{document}
\maketitle
% \newpage

\section{Requirements and Psychology
  and
  The Syntactically Dangerous All and Plural in Specifications
  review}

\subsection{Main contributions}
%You must include a brief description of each paper,
% and an outline of similarities and differences
% between these two papers
Both articles include opinions and suggestions from authors on how to avoid
ambiguous language to make your requirements document more precise, given their
experience in the industry. Although their goals are the same, their approaches
are different. The first article outlines three common defects: ``Deletion'',
``Generalization'', and ``Distortion''. Deletion happens when a verb, or a
so-called ``process-word'' is not defined. For example, the requirement
statement might specify an action like making a report, but who makes the
report, what is reported, when, and how is undefined. Generalization can happen
when universal quatifiers are used, and certain exceptional cases arise, to
which the general statement won't apply. Distortion occurs when the process is
``reformulated into an event''. In the second article, the author focuses on
when and how ``all'', ``each'', and plural words should be used, in order to
minimize ambiguity. Article [2] focuses a bit more on the syntactic nature of
the problems providing examples and, analyzing some math and other languages,
while aritcle [1] focuses more on describing the application of techniques
described.


\subsection{Criticisms} %one paragraph
In both articles [1] and [2], the authors make excellent recommendations overall
on how to improve the preciseness of the language used in requirements
documents, and reducing ambiguous language. However, they both focus very
narrowly on different aspects which can make requirements documents ambiguous.
Unfortunately, following these authors' tips alone will not singlehandedly make
your requirements documents amazingly super precise. For example, the author
from article [2] suggests specific cases where ``all'', ``each'', or plural
words should be used, but while following these recommendations, a statement
which has defects outlined in [1] can still be made. For example, the sentence:
``All intersections must have a single turning signal'' is precise according to
article [2], since ``all'' is used to describe a property of the whole set
(intersections). However, according to article [1], this statement suffers from
generalization, since there might be exceptional cases where a turning signal
may not make sense, or is impossible. The statement can be made more precise by
saying: ``There should be only one turning signal per intersection.'' The
suggestions from the authors, although excellent should be combined with
multiple other techniques for making your requrements document more ideal, that
is with more precice language, and less ambiguities.

\subsection{}

\subsubsection{Difference Between ``Generalization'' and ``All and Plural''}
% Explain, in your own words, the difference between the term “generalization”
% defined in [1] and “all and plural” described in [2]. Give an example
% requirement statement (one sentence) for each of the two phenomena (indicate
% which sentence corresponds to which phenomena), briefly explain what is
% ambiguous about this sentence, and make sure that the examples highlight the
% differences that you explained in the first part of your answer. Do not use
% examples from the articles. Use up to 6 sentences for the answer.

A requirement statement that suffers from the ``All and Plural'' defect can have
multiple meanings, depending on how the reader understands the statement. On the
other hand, requrement statements suffring from the ``Generalization'' defect
tend to have ambiguities resulting from exceptional cases where a certain action
may not be required, or is otherwise impossible.
An example of a atatement with the ``All and Plural'' defect is: ``All traffic
lights at any intersection shall have a single turning signal.'' In this
sentence, one could interpret it as each traffic light having its own respective
turning signal, or someone else could understand it as each intersection having
only one turning signal.
An example requirement with the ``Generalization'' defect is: ``Each traffic
light shall have a left turning signal.`` In this case the ambiguity lies in
whether special cases or exceptions can occur, where there could be situations
such that a left turning signal is not required. For example, an intersection
where left turns are not allowed.

\subsubsection{When Should We Use All vs. Each?}
% Based on the author’s discussion in paper [2], explain when we should use all
% vs. each. Give an illustrative example that would highlight the difference.
% Use up to 4 sentences for the answer.

According to the author in [2] ```all'' should be used when the intention is to
talk about properties of the whole set, whereas ``each'' should be used when the
intention is to talk about properties of each member of the set. For example, we
could say: ``All intersections must have a single turning signal.'' to mean that
there should be only one turning signal per intersection, or we could say:
``Each traffic light at any intersection shall have a turning signal.'' to
convey that every single traffic light should have a turning signal. It can be
noted that these examples suffer from generalization, but that is outside of the
scope of this question.

\subsubsection{Example Requirement Statement}
% Provide an example requirement statement that includes problems related to all
% three defects listed in [1], i.e., deletion, generalization, and distortion.
% Briefly explain how these three defects are implemented in your example. Use
% up to 4 sentences for the answer.

The following requirement statement includes problems related to all three
defects listed in [1], i.e. deletion, generalization, and distortion: ``For all
servers, when a server experiences downtime, a report to the system shall always
occur.'' In this example, deletion is present, since the process of reporting is
ambiguous unless the following are defined beforehand: Who is reporting to whom,
what is being reported, when, and how is it being reported? This example also
contains generalization, as indicated by the use of the universal quantifiers
``all'' and ``always, which may include special cases where the action is not
required, or otherwise impossible. Lastly, the example above also contains
distortion, since it describes an event, which in this case is when a server
experiences downtime. It does not describe how the report is initiated, and how
it ends.


\subsubsection{Do I agree that distortion “often appears in domains with an extensive technical language”?}
% Do you agree with the author of [1] who suggests that distortion “often
% appears in domains with an extensive technical language”? Explain your point
% of view. Give an example requirement statement that includes a distortion and
% uses technical language. Explain why this is a distortion. Use up to 4
% sentences for the answer.

I agree with the author of [1], who suggests that distortion ``often appears in
domains with an extensive technical language''. It is basically natural for a
human to describe a process conditional on some event happening, and it is also
human to forget to elaborate, or to assume that the reader already understands
what the writer is trying to say. An example requirement statement with
distortion is: ``In emergency mode, when the hardware is fixed, the traffic
light system should enter normal mode.'' In this case, the process by which the
system transitions from emergency to normal mode is unknown to the reader from
this statement alone.
%%%%%%%%%%%%%%%%%%%%%%%%%%%%%%%%%%%%%%%%%%%%%%%%%%%%%%%%%%%%%%%%%%%%%%%%%%%%%%%%

\section{Understanding the Customer: What Do We Know
  about Requirements Elicitation? review}

\subsection{Main contributions} % one paragraph
The purpose of this paper is to outline the most common software requrements
elicitation techniques used in the industry. The authors analyzed the overall
trends of multiple studies related to requirements elicitation. According to the
authors, they looked at two papers before, which attempted to do the same thing
before, but those papers pointed out various strategies. Instead, this paper
focuses on one-on-one scenarios, and draws some general conclusions from them.
The paper discusses the benefits and drawbacks of structured and unstructured
interviews, and also explores other non interviewing techniques such as protocol
analysis, ``contrived techniques'' such as card sorting, and textual laddering.
The authors then express what  they found to be the most effective combination
of techniques for requirements elicitation, which is an interview with some
stucture, i.e. some general questions to guide the customer, which can be
combined with some non  interviewing techniques if specific information is
required.


\subsection{Criticisms} %one paragraph

The article does a fantastic job overall of summarizing data from multiple
studies related to requirements elicitation, however, some criticisms can be
made. Although the paper is focused on software requirements elicitation, the
authors of the paper also looked at studies they deemed to be ``related fields''
such as marketing. This raises some questions about the conclusions the authors
are drawing from the data, and how applicable it is to software requirements
elicitation. The article even  explicitly states that ``when we try to abstract
a general conclu- sion from multiple studies, we find that each study has
measured completely dif- ferent effects of the elicitation approaches. This
makes it hard to summarize multiple studies in a straightforward way.'' Although
the authors claim to be doing this to find some broad conclusions, a lot of
detail is lost that may contain some interesting information. The authors
mention that their measurements individually don't give a complete picture, and
that by comparing such a large amount of variables, they've lost the ability to
combine the data statistically, which is arguably one of the more rigorous
approaches for extracting conclusions from the results.


\subsection{}

\subsubsection{Contrived Techniques}
% List the “contrived techniques” that are mentioned by the authors (note that
% they are mentioned in multiple places in the paper). Is card sorting a
% contrived technique? Briefly explain how the author defines the card sorting.
% Use up to 4 sentences for the answer.
The ``contrived techniques`` mentioned by the authors are as follows:
\begin{enumerate}
 \item mock-ups
 \item scenarios
 \item storyboarding
 \item card sorting
\end{enumerate}
The author defines ``contrived techniques'' as a process that asks users to
``engage in some kind of artificial task to help elicit information''. Accoring
to this definition, card sorting is a contrived technique, since the user is
asked to arrange some concepts or entities from the problem domain into groups,
which fits the description of an ``artificial task'' that helps with the
elicitation of information.

\subsubsection{Unstructured vs. Structured Interviews}
% According to the authors, is it true that unstructured interviews are
% preferred than the structured interviews? What is the main benefit, which was
% mentioned by the authors, of using a structured interview? Use up to 3 sentences for the answer.

According to the authors, ``interviews with some stucture are preferable to
completely unstructures ones.'' In their view, the main benefit of  structured
interviews is that more customer needs tend to be identified during structured
interviews. They also mention that the experience of the developers doing the
interviewing did not affect the amount of customer needs identified.


\subsubsection{Main Technique for Elicitation of Requirements}
% According to the authors, what is the main technique for elicitation of
% requirements? Can the contrived techniques be used during an interview? Use up
% to 2 sentences for the answer.

According to the authors, the main technique for requirements elicitation is
interviewing. They suggest that non interviewing techniques, including
``contrived Techniques'' can be used to improve the efficiency of
a structured interview, when specific types of information are
required.










\end{document}
