\documentclass[letterpaper,12pt]{article}
\synctex=1

\usepackage[margin=1in]{geometry}
% \usepackage{lipsum}
%
% \usepackage{hyperref}
%
% \usepackage{graphicx}


% \bibliographystyle{IEEEtran}

\title{ECE 321: Software Requirements Engineering \\ Assignment 1}
\author{Arun Woosaree \\ \\ XXXXXXX}


\begin{document}
\maketitle

\section{Requirements and Psychology
  and
  The Syntactically Dangerous All and Plural in Specifications
  review}

\section{Main contributions}
%You must include a brief description of each paper,
% and an outline of similarities and differences
% between these two papers


\section{Criticisms} %one paragraph

\section{}

\subsection{Difference Between ``Generalization'' and ``All and Plural''}
% Explain, in your own words, the difference between the term “generalization”
% defined in [1] and “all and plural” described in [2]. Give an example
% requirement statement (one sentence) for each of the two phenomena (indicate
% which sentence corresponds to which phenomena), briefly explain what is
% ambiguous about this sentence, and make sure that the examples highlight the
% differences that you explained in the first part of your answer. Do not use
% examples from the articles. Use up to 6 sentences for the answer.

\subsection{When Shouls We Use All vs. Each?}
% Based on the author’s discussion in paper [2], explain when we should use all
% vs. each. Give an illustrative example that would highlight the difference.
% Use up to 4 sentences for the answer.

\subsection{Example Requirement Statement}
% Provide an example requirement statement that includes problems related to all
% three defects listed in [1], i.e., deletion, generalization, and distortion.
% Briefly explain how these three defects are implemented in your example. Use
% up to 4 sentences for the answer.

\subsection{Do I agree that distortion “often appears in domains with an extensive technical language”?}
% Do you agree with the author of [1] who suggests that distortion “often
% appears in domains with an extensive technical language”? Explain your point
% of view. Give an example requirement statement that includes a distortion and
% uses technical language. Explain why this is a distortion. Use up to 4
% sentences for the answer.

%%%%%%%%%%%%%%%%%%%%%%%%%%%%%%%%%%%%%%%%%%%%%%%%%%%%%%%%%%%%%%%%%%%%%%%%%%%%%%%%

\section{Understanding the Customer: What Do We Know
  about Requirements Elicitation? review}

\section{Main contributions} % one paragraph 


\section{Criticisms} %one paragraph

\section{}

\subsection{Contrived Techniques}
% List the “contrived techniques” that are mentioned by the authors (note that
% they are mentioned in multiple places in the paper). Is card sorting a
% contrived technique? Briefly explain how the author defines the card sorting.
% Use up to 4 sentences for the answer.

\subsection{Unstructured vs. Structured Interviews}
% According to the authors, is it true that unstructured interviews are
% preferred than the structured interviews? What is the main benefit, which was
% mentioned by the authors, of using a structured interview? Use up to 3 sentences for the answer.


\subsection{Main Technique for Elicitation of Requirements}
% According to the authors, what is the main technique for elicitation of
% requirements? Can the contrived techniques be used during an interview? Use up
% to 2 sentences for the answer.










\end{document}
