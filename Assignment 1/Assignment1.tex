\documentclass[letterpaper,12pt]{article}
\synctex=1

\usepackage[margin=1in]{geometry}
% \usepackage{lipsum}
%
% \usepackage{hyperref}
%
% \usepackage{graphicx}


% \bibliographystyle{IEEEtran}

\title{ECE 321: Software Requirements Engineering \\ Assignment 1}
\author{Arun Woosaree \\ \\ XXXXXXX}


\begin{document}
\maketitle

\section{``Passing the word'' Review}

\subsection{Main Contributions}

The author mainly goes over his recommendations  for best practice in the
industry, based on his past experiences. He begins with describing the
characteristics of an ideal manual in his eyes. The article also gives various
tips for writing manuals, such as making the manual's language more consistent.
The author also describes the importance of the formal definition. He also
mentions the usefulness of having informal documentation, and how it might be
created. Many other tips are outlined in the article as well. The author
explains methods used, and the structure of meetings that he found to work most
effectively in his experience. The author also outlines the importance of having
multiple implementations in a product's development cylce, and how it relates to
the manual.  Finally, the author goes over the importance of product testing,
which brings everything mentioned in the article together and serves as a final
checkpoint before the product is ready for the customer.

\subsection{Criticisms}
The author makes mostly good points in this article. Naturally, there are some
points I must disagree with, the first being the author's opinion that the
manual be written by only one or two men. This approach is simply not scalable,
and is impractical for large projects. Furthermore, having multiple people
collaborate on the manual provides the opportunity for people with multiple
perspectives to review and edit the document, so that inaccuracies can be found
and fixed quicker. Having multple perspectives also makes it easier to identify
portions of the document which may need to be reworded to make it easier or more
interesting to read. Additionally, the style in which one person writes one week
might be different in the next week, based on that person's emotions. With multiple
writers, each writer can keep a lookout on each other to make sure the language
is consistent. Next, I must disagree with the author's suggestions regarding
meetings. While his suggestions may have been effective years ago, nowadays, the
agile workflow seems more attractive. Meetings are held every day between the
people developing the product, but they are short. This way, problems can be
found quickly and it allows the team to prioritize issues effectively, in
addition to making the right people aware of issues that may concern them. My
suggestion for a second type of meeting would be before any major product
deployment, and the frequency and length of these meetings should depend on the
scale of the project. If deployments are happening frequently, this might be a
monthly affair, or an annual event like the author suggested. This meeting
should consist of the people developing the product as before, and also any
managers, representatives, product owners, stakeholders, etc.

\subsection{Answers to questions}
\subsubsection{Good manual style}

In the author's view, a well-written manual must use precise, consistent
language, have quantized changes, and  it should also describe every detail the
user sees, omitting the details that the user does not see. The author mentions
that this can be done in a formal and/or inform
al style. In my opinion, an
informal style is more desireable, since quick ammendments and changes can be made to
the document as necessary, and using informal language makes it easier and less
daunting to read overall for the average person.

\subsubsection{Effective meetings}

The first type of meeting the author claims is effective is a weekly half-day
conference led by the chief architect, as well as all the architects, official
representatives of the hardware and software implementers, and market planners.
The second type of meeting is what the author refers to as an annual ``supreme
court session'', which lasts two weeks. In this type of meeting, the project
manager leads, with members from the architecture group,  the programmers' and
implementers' architectural representatives, and the managers of programming,
marketing, and im plementation efforts joining.

\subsubsection{Independent product-testing organization}

In the author's view, an independent product-testing organization acts  as a
``surrogate customer'', which allows for early detection of bugs and departure
from the design. I mostly agree with this viewpoint, since programmers are
human, and are prone to making mistakes like everyone else. Although it might
slow development time, having a dedicated team for finding these flaws before
deployment is important, so that the customer has a better overall experience.

\section{``Requirements Engineering: The State of the Practice Review''}

\subsection{Main Contributions}
The purpose of this paper is to bring to light the most common practices
and techniques used in the software development industry. According to
the authors, before this paper, most people including themselves were
referring to sources without credible facts or statistics to back their claims.
This paper aims to fix that. The authors have taken the time to
survey a group of people who they felt would be representative of the
software development industry, and to compile the results to make a few graphs.
The paper seems to accomplish its goal of providing statistics and facts, as well as
analysis of the data in a format that is credible and can be cited by others who
need a source to back up their claims about common practices in the software industry.

\subsection{Criticisms}
The paper does a fantastic job overall in achieving its goal. However, some criticisms
can be made about the design of the survey created by the authors.
The authors based their survey on two surveys done by different Universities.
Perhaps it could have been designed differently from these already conducted surveys,
to potentially reveal some new and / or different  data.  Secondly, while the
sample population in their survey is considerably larger than the ones in the surveys from the
two univerisities, 194 respondents is still not a lot of people, especially considering that the
software development industry is one of the biggest in the world. Although the
sample population was quite diverse, and care was taken to ensure reliable data (such as
making sure that the participants are all industrial practitioners), it is questionable
how useful the data collected is, given the small population size.


\subsection{Answers to questions}

\subsubsection{Most frequently used lifecycle model}

According to the survey, the Incremental lifecycle model is most popular for projects which last over
two years, while the Waterfall model is more widely used for shorter projects.
The Incremental model is likey most popular for longer projects, because
the project may evolve and necessitate changes as time passes, or as the
stakeholders may change preferences in that time. On the other hand,
shorter projects are less likely to undergo much change due to their shorter
lifecycle, so in this case, the waterfall model is ideal because the end result is
agreed upon before the project begins, which can be more efficient compared
to other models.

% Because the stages
% are very clearly outlined, and the end result is agreed upon before the project
% begins in the waterfall model, which can result in an end product sooner than
% other models.

\subsubsection{Prototyping}

Of the respondents in the survey, about 60?\% did some sort of prototyping.
The most frequent type of prototyping is with the user interface. This is likely
because the user interface is what's visible in the end product, and is what the
stakeholders are most likely concerned with, as opposed to the technical implementation.
\subsubsection{Elicitation of users' requirements}

According to the survey, the top three methods for
elicitation of users’ requirements are: Scenarios \& use cases, focus groups, and informal modelling.
Given the option, I would probably choose to the use cases and scenarios approach for requirements
elicitation. Use cases and scenarios are an attractive option, because the focus is on the end goal,
which does not need to be technical. This lack of technicality can simplify communications with stakeholders,
since the stakeholder does not need to know the technical details of the project, just how it should work.
This can help mitigate problems where the end product is drastically different from what the stakeholder(s)
imagined, which explains why it's the most widely used technique.

\subsubsection{Is the informal modelling of the requirements more
 popular than the formal representations}

According to the survey, formal models are rarely used, and informal models are
much more popular, which is likely because informal models make it easier and more
natural to work with. This may end up in a higher quality end product, as less time
is spent on obsessing over unessecary details which don't affect the end product.
\subsubsection{According to the survey, were longer projects less often finished on
 time and within the budget?}
Yes.
% When we examine the responses with re-
% spect to the project duration, however, we see
% considerably less parity. Indeed, when projects
% run longer than two years, managerial and
% technical staff believe that they keep to sched-
% ule and budget only about 20 percent of the
% time, compared to 60 percent of projects last-
% ing up to one year. Responses for projects last-
% ing beyond two years were, on the whole,
% more negative.


















\end{document}
