\documentclass[letterpaper,12pt]{article}
\synctex=1

\usepackage{geometry}
\usepackage{lipsum}

\usepackage{hyperref}

\usepackage{graphicx}


% \bibliographystyle{IEEEtran}

\title{ECE 321: Software Requirements Engineering \\ Assignment 1}
\author{Arun Woosaree \\ \\ XXXXXXX}


\begin{document}
\maketitle

\section{``Passing the word'' Review}

\subsection{Main Contributions}
\begin{itemize}
 \item describes the importance of a manual, or written specification
 \item describes good manual style
 \item formal definitions
 \item the importance of meetings
 \item importance of multipel implementations
 \item product testing
\end{itemize}

\subsection{Criticisms}
Shouldn't be too bad...

\subsection{Answers to questions}
\subsubsection{Good manual style}
\begin{itemize}
 \item It describes  and  prescribes  every  detail  of  what  the  user sees
 \item it refrains from describing things the user does not see
 \item For  the  sake of  implementers  it is  importantthat  thechanges  bequantized—thatthere  be  dated versionsappearing  on a  schedule
 \item consistency
 \item precise language
\end{itemize}
\subsubsection{Effective meetings}
\begin{itemize}
 \item The  first is  a  weekly  half-day conference  of  all  the  architects,
       plus official representatives  of the  hardware  and  software implementers,
       and the  market planners. The chief  system architect presides. Anyone  can
       propose  problems  or  changes,  but  proposals  are usually distributed in
       writing before the  meeting. A new problem is  usually discussed  a while. The
       emphasis  is  on creativity, rather than merely decision. The group attempts
       to  invent many solu- tions  to problems, then a few solutions  are  passed  to
       one  or  more of the architects for detailing into precisely worded manual
       change proposals. Detailed change proposals then come up for decisions. These
       have  been circulated  and carefully considered  by implementers and users, and
       the pros and cons are well delineated. If a consensus emerges, well and good.
       If not, the chief architect decides. Minutes Conferences and  Courts §7 are
       kept  and  dec isions  are formally, promptly,  and  widely  dis- seminated.
       Decisions from the weekly conferences give quick results and allow work to
       proceed. If anyone is too unhappy, instant appeals to the project manager are
       possible, but this happens very rarely. The  fruitfulness  of these  meetings
       springs from several sources: 1. The same group—architects, users, and
       implementers—meets weekly for months. No time is needed for bringing people up
       to  date. 2. The group is bright, resourceful, well versed in the issues, and
       deeply  involved  in  the  outcome. No  one  has  an "advisory" role. Everyone
       is authorized to make binding commitments. 3. When  problems  are raised,
       solutions ar e sought both within and  outside  th e obvious boundaries. 4. The
       formality  of written proposals  focuses attention, forces decision, and avoids
       committee-drafted inconsistencies. 5. The  clear  ve sting of decision-making
       power in the chief archi- tect  avoids compromise and delay.

 \item annual supreme court sessions, lasting typically two weeks. (I would hold
       them every six months if I were doing it again.) These sessions were held just
       before major freeze dates for the manual. Those present included not only the
       architecture group and the programmers' and implementers' arc hitectural
       representa- tives, but also the manag ers of programming, marketing, and im-
       plementation efforts. The System/360 project manager presided. The agenda
       typically consisted of about 200 items, mostly minor, which were enumerated in
       charts placarded around the room. All 68 Passing  the  Word sides were heard
       and decis ions made. By the miracle of computer- ized  text editing  (and lots
       of  fine staff work), each  participant found an updated manual, embodying
       yesterday's decisions, at his seat every  morning. These  "fall festivals" were
       useful not only fo r resolving deci- sions, but  also  for getting  them
       accepted. Everyone  was  heard, everyone participated, everyone understood
       better  the intricate constraints  and  interrelationships  among  de cisions.

\end{itemize}
\subsubsection{Independent product-testing organization}

In the author's view, an independent product-testing organization acts  as a
``surrogare customer'', which allows for early detection of bugs and departure
from the design. I mostly agree with this viewpoint, since programmers are
human, and are prone to making mistakes like everyone else. Although it might
slow development time, having a dedicated team for finding these flaws  before
deployment is important, so that the customer has a better overall experience.

\section{``Requirements Engineering: The State of the Practice Review''}

\subsection{Main Contributions}
\lipsum[3]

\subsection{Criticisms}
\lipsum[4]

\subsection{Answers to questions}
\subsubsection{Most frequently used lifecycle model}
\lipsum[66]
\subsubsection{Prototyping}
\lipsum[66]
\subsubsection{Elicitation of users' requirements}
\lipsum[75]
\subsubsection{Informal modelling of requirements}
"Lorem ipsum dolor sit amet, consectetur adipiscing elit, sed do eiusmod tempor incididunt ut labore et dolore magna aliqua. Ut enim ad minim veniam, quis nostrud exercitation ullamco laboris nisi ut aliquip ex ea commodo consequat.
\subsubsection{Are longer projects less often finished on time and within budget?}
Yes/No one sentence answer ok cool.

















\end{document}
