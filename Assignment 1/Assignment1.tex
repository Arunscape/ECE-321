\documentclass[letterpaper,12pt]{article}
\synctex=1

\usepackage{geometry}
\usepackage{lipsum}

\usepackage{hyperref}

\usepackage{graphicx}


% \bibliographystyle{IEEEtran}

\title{ECE 321: Software Requirements Engineering \\ Assignment 1}
\author{Arun Woosaree \\ \\ XXXXXXX}


\begin{document}
\maketitle

\section{``Passing the word'' Review}

\subsection{Main Contributions}
\begin{itemize}
 \item describes the importance of a manual, or written specification
 \item describes good manual style
 \item formal definitions
 \item the importance of meetings
 \item importance of multipel implementations
 \item product testing
\end{itemize}

\subsection{Criticisms}
The author makes mostly good points in this article. Naturally, there are a few
points I must disagree with, the first being the author's opinion that the
manual be written by only one or two men. This approach is simply not scalable,
and is impractical for large projects. Furthermore, having multiple people
collaborate on the manual provides the opportunity for people with multiple
perspectives to review and edit the document, so that inaccuracies  can be found
and fixed quicker. Having multple perspectives also makes it easier to identify
portions of the document which may need to be reworded to make it easier or more
interesting to read. Additionally, the style in which one person writes one week
might be different the next, based on that person's emotions. With multiple
writers, each writer can keep a lookout on each other to make sure the language
is consistent. Next, I must disagree with the author's suggestions regarding
meetings. While his suggestions may have been effective years ago, nowadays, the
agile workflow seems more attractive. Meeting are held every day between the
people developing the product, but they are short. This way, problems can be
found quickly and  it allows the team to prioritize issues effectively, in
addition to making the right people aware of issues that may concern them. My
suggestion for a second type of meeting would be before any major product
deployment, and the frequency and length of these meetings should depend on the
scale of the project. If deployments are happening frequently, this might be a
monthly affair, or an annual event like the author suggested. This meeting
should consist of the people developing the product as before, and also any
managers, representatives, product owners, stakeholders, etc.

\subsection{Answers to questions}
\subsubsection{Good manual style}

In the author's view, a well-written manual must use precise, consistent
language, have quantized changes, and  it should also describe every detail the
user sees, omitting the details that the user does not see. The author mentions
that this can be done in a formal and/or informal style. In my opinion, an
informal style is more desireable, since quick ammendments and changes can be made to
the document as necessary, and using informal language makes it easier and less
daunting to read overall for the average person.

\subsubsection{Effective meetings}

The first type of meeting the author claims is effective is a weekly half-day
conference led by the chief architect, as well as all the architects, official
representatives of the hardware and software implementers, and market planners.
The second type of meeting is what the author refers to as an annual ``supreme
court session'', which lasts two weeks. In this type of meeting, the project
manager leads, with members from the architecture group,  the programmers' and
implementers' architectural representatives, and the managers of programming,
marketing, and im plementation efforts joining.

\subsubsection{Independent product-testing organization}

In the author's view, an independent product-testing organization acts  as a
``surrogate customer'', which allows for early detection of bugs and departure
from the design. I mostly agree with this viewpoint, since programmers are
human, and are prone to making mistakes like everyone else. Although it might
slow development time, having a dedicated team for finding these flaws before
deployment is important, so that the customer has a better overall experience.

\section{``Requirements Engineering: The State of the Practice Review''}

\subsection{Main Contributions}
\lipsum[3]

\subsection{Criticisms}
\lipsum[4]

\subsection{Answers to questions}
\subsubsection{Most frequently used lifecycle model}
\lipsum[66]
\subsubsection{Prototyping}
\lipsum[66]
\subsubsection{Elicitation of users' requirements}
\lipsum[75]
\subsubsection{Informal modelling of requirements}
"Lorem ipsum dolor sit amet, consectetur adipiscing elit, sed do eiusmod tempor incididunt ut labore et dolore magna aliqua. Ut enim ad minim veniam, quis nostrud exercitation ullamco laboris nisi ut aliquip ex ea commodo consequat.
\subsubsection{Are longer projects less often finished on time and within budget?}
Yes/No one sentence answer ok cool.

















\end{document}
